\documentclass{sig-alternate-05-2015}
\usepackage[l2tabu,orthodox]{nag}
\usepackage[utf8x]{inputenc}       
\usepackage[british]{babel}        
\usepackage[babel=true]{microtype}
\usepackage{amsmath}
\usepackage[all]{onlyamsmath}
\usepackage{newtxtext}
\usepackage{newtxmath}
\usepackage{upquote}
\usepackage{graphicx}
\usepackage{url}
\usepackage[caption=false]{subfig}
\usepackage{booktabs}
\usepackage{bytefield}
\usepackage{listings}
\usepackage{algorithm}
\usepackage{algpseudocode}
\usepackage{color}
\usepackage{balance}

\newcommand{\todo}[1]{\textbf{\textcolor{red}{To do: #1}}}

%==================================================================================================
% The following information gets written into the PDF file information:
\pdfinfo{
  /Title        (Implementing Real-time Transport Services over an Ossified Network)
  /Author       (Stephen McQuistin, Colin Perkins, and Marwan Fayed)
  /Subject      (Real-time Transport Protocols)
  /Keywords     (Transport Services, TCP, TCP Hollywood)
  /CreationDate (D:20160516000000+01'00')
  /ModDate      (D:20160516000000+01'00')
  /Creator      (LaTeX)
  /Producer     (pdfTeX)
}
%==================================================================================================
\begin{document}
\title{Implementing Real-time Transport Services over an Ossified Network}
\numberofauthors{3}
\author{
  \alignauthor
    Stephen McQuistin\\
    \affaddr{University of Glasgow, UK}\\
    \email{sm@smcquistin.uk}
  \alignauthor
    Colin Perkins\\
    \affaddr{University of Glasgow, UK}\\
    \email{csp@csperkins.org}
  \alignauthor
    Marwan Fayed\\
    \affaddr{University of Stirling, UK}\\
    \email{mmf@cs.stir.ac.uk}
}

\toappear{}
\maketitle
%==================================================================================================
\begin{abstract}

% Four sentences:
%  - State the problem
%  - Say why it's an interesting problem
%  - Say what your solution achieves
%  - Say what follows from your solution



\end{abstract}
%==================================================================================================
\begin{CCSXML}
  <ccs2012>
    <concept>
      <concept_id>10003033.10003039.10003040</concept_id>
      <concept_desc>Networks~Network protocol design</concept_desc>
      <concept_significance>500</concept_significance>
    </concept>
    <concept>
      <concept_id>10003033.10003039.10003048</concept_id>
      <concept_desc>Networks~Transport protocols</concept_desc>
      <concept_significance>500</concept_significance>
    </concept>
  </ccs2012>
\end{CCSXML}

\ccsdesc[500]{Networks~Network protocol design}
\ccsdesc[500]{Networks~Transport protocols}

\printccsdesc

\keywords{Transport protocols; real-time multimedia applications}
%==================================================================================================
\section{Introduction}

% A good paper introduction is fairly formulaic. If you follow a simple set
% of rules, you can write a very good introduction. The following outline can
% be varied. For example, you can use two paragraphs instead of one, or you
% can place more emphasis on one aspect of the intro than another. But in all
% cases, all of the points below need to be covered in an introduction, and
% in most papers, you don't need to cover anything more in an introduction.
%
% Paragraph 1: Motivation. At a high level, what is the problem area you
% are working in and why is it important? It is important to set the larger
% context here. Why is the problem of interest and importance to the larger
% community?

Real-time applications are increasingly important in the Internet. 
We want to make it easier to write these applications, while improving
the quality of experience for users of those applications by lowering
latency and increasing robustness. 
In this, we are challenged by the limitations of the standard Internet 
transport protocols, and by the ossified nature of the network, which makes
it increasingly difficult to deploy new transports. 

% Paragraph 2: What is the specific problem considered in this paper? This
% paragraph narrows down the topic area of the paper. In the first
% paragraph you have established general context and importance. Here you
% establish specific context and background.

In practice, the only protocols that can be widely used in the Internet are
TCP and UDP. Other protocols are blocked by firewalls and other middleboxes.  
TCP provides sophisticated congestion control, coupled with a reliable,
ordered, byte stream API. These are suitable for many applications, but are
not necessarily appropriate for real-time traffic.  UDP exposes the best-effort 
IP packet delivery service, offering flexibility to develop new protocols,
but at the cost of requiring completely new protocol mechanisms to be
defined. Both protocols are usable for real-time applications, but neither
really provides the right services and API.  This forces each application
to re-invent mechanisms that should be provided by the transport, increasing 
costs and complexity, raising barriers to innovation.

% Paragraph 3: "In this paper, we show that...". This is the key paragraph
% in the introduction - you summarize, in one paragraph, what are the main
% contributions of your paper, given the context established in paragraphs 
% 1 and 2. What's the general approach taken? Why are the specific results
% significant? The story is not what you did, but rather:
%  - what you show, new ideas, new insights
%  - why interesting, important?
% State your contributions: these drive the entire paper.  Contributions
% should be refutable claims, not vague generic statements.

In this paper we discuss the appropriate set of transport services and APIs
for real-time applications.
We demonstrate that it is possible to realise these services and APIs
in the context of both TCP and UDP, despite the limitations imposed by the
legacy protocols, middleboxes, and ossification of the network. 
We present initial results to show that, despite the observed ossification,
the network has the flexibility to deploy new transport protocols, if care
is taken to reinterpret layer boundaries and deploy those new transports in
the context of TCP and UDP.

% Paragraph 4: What are the differences between your work, and what others
% have done? Keep this at a high level, as you can refer to future sections
% where specific details and differences will be given, but it is important
% for the reader to know what is new about this work compared to other work
% in the area.

Our contributions are to identify the needs of real-time applications, and
the appropriate transport services and APIs to support those needs; 
to illustrate how those transport services can be realised on the current
Internet, in the context of UDP and TCP deployments; and 
to present some initial measurement results to show that the mechanisms we
propose ought to be usable in the public Internet. 

% Paragraph 5: "We structure the remainder of this paper as follows." Give
% the reader a road-map for the rest of the paper. Try to avoid redundant
% phrasing, "In Section 2, In section 3, ..., In Section 4, ... ", etc.

We begin, in Section \ref{sec:services} by discussing transport services 
for real-time applications, and outlining the common conceptual API that
those applications use. 
This is followed, in Section \ref{sec:ossification}, by a review of the
deployment considerations for new protocols, caused by ossification of
the network, with Section \ref{sec:partial} considering, in particular, how
TCP reliability semantics can evolve within the constraints of the deployed
infrastructure.
Section \ref{sec:realising} outlines how the transport services we have
identified can be realised in practical networks.
Finally, Section \ref{sec:related} discusses related work, and Section
\ref{sec:conclusions} concludes.

%==================================================================================================
\section{Real-time Transport Services}
\label{sec:services}

In the IETF, the Transport Services (TAPS) working group is chartered to 
(1) develop a taxonomy of \emph{transport services}, i.e., to identify the
features that comprise, and can be combined, to form complete transport
protocols; and (2) to develop an abstract API for applications to request
desirable services, allowing the system to select an appropriate transport
protocol based on application needs. It is hoped that this will loosen the
coupling between application and transport, so enabling deployment of new 
transport protocols.

...TAPS is valuable, because it gives us a vocabulary for discussing the
   components of transport protocols, that we can use to identify the needs
   of real-time applications

...the key feature of such applications is clearly timing. That is, since
   these applications convey real-time data, they all have some concept 
   of a \emph{deadline} by which that data must be presented, after which it 
   becomes useless. 
   the deadline might be short or long. Interactive applications, such as
   telephony, video conferencing, or telepresence, require low end-to-end
   latency, with the deadline for presenting data being tens, or perhaps
   low hundreds, of milliseconds after it was generated. Non-interactive
   applications, such as broadcast TV, may have deadlines in the order of
   seconds, or even tens of seconds for on-demand programming. 
   These deadlines are unusual for real-time systems, in that they are
   simultaneously flexible and strict: flexible in that the exact value
   of the deadline is typically not important, provided it is of the 
   right order-of-magnitude for the application, but strict in that 
   any particular deadline provides a hard cut-off, after which the
   data is useless. 

...the presence of deadlines, in a network that offers only a best-effort
   packet deliver service, implies that the transport must provide only
   \emph{partial reliability}.
   the network must be assumed to have non-zero probability of losing any
   particular packet. If FEC is used to repair loss, this implies there is
   always some probability that a particular packet will be non-recoverable. 
   If retransmission is used to repair loss, this implies that there is
   always the potential of unbounded delay. 
   These probabilities can be estimated and bounded, but will be non-zero.
   Accordingly, to meet deadlines, the transport must offer a partial
   reliability service. 
   Many real-time applications run over TCP today, and TCP does not offer
   this service. This results in occasional play-out stalls, when the
   transport blocks the application. These are one of the primary causes
   of poor user experience in streaming applications, and this is directly
   caused by the lack of partial reliability. Missing one frame that has
   not been delivered by its deadline is much less disruptive than a stall
   in play-out. 

...given deadlines and partial reliability, the next important transport
   service is \emph{dependency management}. There is no point sending
   data that cannot be used, because it depends on previous data that
   was lost. 
   complex trade-off, since you might want to send data that misses its 
   deadline, if it allows later data to be used. 

...following from this, we have the concept of application-level framing
   to make the best use of data, irrespective of losses. This implies that
   the transport provides a \emph{message oriented} service that respects
   ADU boundaries. 

...we note that many multimedia applications make use of multiple streams. 
   For example, audio and video are logically separate, etc.
   can be supported by opening multiple transport connections, but
   desirable to support a \emph{sub-stream} service to reduce overheads.

...finally, we note that \emph{congestion control} is important. 
   motivate: historically there has been a belief that real-time doesn't 
   do congestion control, needs an isochronous channel. 
   not practical for the Internet. 
   Also, experience has shown not necessary for many real-time applications.
   Yes, there are some that need constant bandwidth and cannot adapt, but
   many are adaptive. 
   Better user experience to adapt to fit the channel, rather than recover
   from loss after the channel discards packets that don't fit. 


..what a conceptual API for this would look like?

%==================================================================================================
\section{Innovation and Ossification}
\label{sec:ossification}

Briefly discuss: the Internet architecture, in principle, allows free
innovation at the transport layer, provided the IP layer is unchanged. 
Outline why this is not true in practice. 

Implication: new protocols must look like TCP or UDP.

UDP is the obvious choice, since it provides minimal additional services
over the IP layer, allowing great flexibility in innovation for protocols
tunnelled on top. Only real cost is a few bytes of additional header.

But, UDP is often blocked. Corporate firewalls. History of abuse for DDoS
attacks (draft-byrne-opsec-udp-advisory-00). Slowly changing, e.g., due to
QUIC, WebRTC, but not clear that UDP is universally available.

TCP is a more complex choice for innovation. More sophisticated protocol.
Complex headers. Mandates much more behaviour. 

However:
- end-points can change congestion control, provided it's ACK-based
- segments need be sent with contiguous sequence number space, resending
  segments to fill gaps in sequence number space caused by packet loss
- traditional API is restrictive, but flexibility to do out-of-order 
  delivery to applications
- easy to perform consistent segmentation on sending, and add COBS framing, 
  to get message-oriented semantics

Main issue with adding what we need is reliability. 
Leads to the question: can we work-around TCP reliability?

%==================================================================================================
\section{Partial Reliability and TCP}
\label{sec:partial}

approach: 
- separate out sequence numbers in sent segments with sequence numbers
  visible to the application.
- COBS encode for message boundaries.
- retransmission have same segment sequence number, but different contents
  and application sequence numbers

Does it work?
- Initial deployment results from the UK
- Is there any more data that can be got quickly?

Any other surveys showing the same thing?

%==================================================================================================
\section{Realising Transport Services}
\label{sec:realising}

Assuming the above is true more generally, once wider measurements have
been conducted...

here is an abstract API for real-time transport services

we can clearly layer it above UDP: the services look a lot like PR-SCTP,
and experiences with the WebRTC data channel show that this can run over
UDP and be deployed in the wild.

these services are deployed over TCP now, but have an inconvenient API that
imposes lots of work on application developers, and with higher than desired
latency. 
TCP Hollywood offers a way to fix both issues.
Brief summary.
Obviously we need to change TCP congestion control too.

%==================================================================================================
\section{Related Work}
\label{sec:related}

% This should come near the end, and focussing on discussing how your work
% relates to that of others. Any relevant related work should have been
% cited already, so this is not a list of related work, it's a discussion
% of how that work relates.
%
% Why not put related work after the introduction? 1) because describing
% alternative approaches gets between the reader and your idea; and 2)
% because the reader knows nothing about the problem yet, so your
% (carefully trimmed) description of various technical trade-offs is
% absolutely incomprehensible.
% 
% When writing the related work:
%  - Give credit to others where it's due; this doesn't diminish the
%    credit you get from your paper. 
%  - Acknowledge weaknesses in your approach.
%  - Ensure related work is accurate and up-to-date



%==================================================================================================
\section{Conclusions}
\label{sec:conclusions}

to make effective use of the network, need to deploy new transport services
and protocols

seems likely that the appropriate long-term approach for doing this is to
repurpose UDP as a demultiplexing layer for higher-layer protocols. 

in the short term, we need to deal with networks that block UDP, and be
able to run over TCP.
our initial work shows that this is frequently possible, and works well
enough to offer benefit.

%==================================================================================================
\section{Acknowledgements}

% Acknowledge funding sources.

%==================================================================================================
\bibliographystyle{abbrv}
\bibliography{paper}
\end{document}
% vim: set ts=2 sw=2 tw=75 et ai:
