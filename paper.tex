\documentclass{sig-alternate-05-2015}
\usepackage[l2tabu,orthodox]{nag}
\usepackage{fixltx2e}
\usepackage[utf8x]{inputenc}       
\usepackage[british]{babel}        
\usepackage[babel=true]{microtype}
\usepackage{amsmath}
\usepackage[all]{onlyamsmath}
\usepackage{newtxtext}
\usepackage{newtxmath}
\usepackage{upquote}
\usepackage{graphicx}
\usepackage{url}
\usepackage[caption=false]{subfig}
\usepackage{booktabs}
\usepackage{bytefield}
\usepackage{listings}
\usepackage{algorithm}
\usepackage{algpseudocode}
\usepackage{color}
\usepackage{balance}

\newcommand{\todo}[1]{\textbf{\textcolor{red}{To do: #1}}}

%==================================================================================================
% The following information gets written into the PDF file information:
\pdfinfo{
  /Title        (...)
  /Author       (...)
  /Subject      (...)
  /Keywords     (..., ..., ...)
  /CreationDate (D:20150827110616+01'00')
  /ModDate      (D:20150827110616+01'00')
  /Creator      (LaTeX)
  /Producer     (pdfTeX)
}
%==================================================================================================
\begin{document}
\title{Implementing Real-time Transport Services over an Ossified Network}
\numberofauthors{3}
\author{
  \alignauthor
    Stephen McQuistin\\
    \affaddr{University of Glasgow, UK}\\
    \email{sm@smcquistin.uk}
  \alignauthor
    Colin Perkins\\
    \affaddr{University of Glasgow, UK}\\
    \email{csp@csperkins.org}
  \alignauthor
    Marwan Fayed\\
    \affaddr{University of Stirling, UK}\\
    \email{mmf@cs.stir.ac.uk}
}

\toappear{}
\maketitle
%==================================================================================================
\begin{abstract}

% Four sentences:
%  - State the problem
%  - Say why it's an interesting problem
%  - Say what your solution achieves
%  - Say what follows from your solution



\end{abstract}
%==================================================================================================
\begin{CCSXML}
  <ccs2012>
    <concept>
      <concept_id>10003033.10003039.10003040</concept_id>
      <concept_desc>Networks~Network protocol design</concept_desc>
      <concept_significance>500</concept_significance>
    </concept>
    <concept>
      <concept_id>10003033.10003039.10003048</concept_id>
      <concept_desc>Networks~Transport protocols</concept_desc>
      <concept_significance>500</concept_significance>
    </concept>
  </ccs2012>
\end{CCSXML}

\ccsdesc[500]{Networks~Network protocol design}
\ccsdesc[500]{Networks~Transport protocols}

\printccsdesc

\keywords{Transport protocols; real-time multimedia applications}
%==================================================================================================
\section{Introduction}

% General advice:
%  - Avoid stock and cliche phrases such as "recent advances in XYZ" or
%    anything alluding to the growth of the Internet (unless your paper
%    is about the growth of the Internet!).
%  - Be sure that the introduction lets the reader know what this paper is
%    about, not just how important your general area of research is.
%    Readers won't stick with you for three pages to find out what you are
%    talking about.
%  - The introduction must motivate your work by pinpointing the problem
%    you are addressing and then give an overview of your approach and/or
%    contributions (and perhaps even a general description of your
%    results).  In this way, the intro sets up my expectations for the
%    rest of your paper -- it provides the context, and a preview.
%  - Repeating the abstract in the introduction is a waste of space.

% para. 1: motivation: broadly, what is problem area, why important? 



% para. 2: narrow down: what is problem you specifically consider



% para. 3: "In this paper, we ...": most crucial paragraph, tell your
% elevator pitch. The story is not what you did, but rather:
%  - what you show, new ideas, new insights
%  - why interesting, important?
% State your contributions: these drive the entire paper.  Contributions
% should be refutable claims, not vague generic statements.

In this paper, we ...

% para. 4: how different/better/relates to other work



% para. 5: "We structure the remainder of this paper as follows."

We structure the remainder of this paper as follows.

%==================================================================================================
% Concentrate single-mindedly on a narrative that:
%  - Describes the problem, and why it's interesting
%  - Describes your idea
%  - Defends your idea, showing how it solves the problem, and filling out
%    the details
% On the way, cite relevant work in passing, but defer discussion to the
% end.
%
% Introduce the problem, and your idea, using examples, and only then
% present the general case. Explain the idea as if your were speaking to
% someone using a whiteboard. Conveying the intuition is primary; details
% follow. Write in a top down manner: state broad themes and ideas first,
% then go into details.
%
% The introduction makes claims. The body of the paper provides evidence
% to support each claim. Check each claim in the introduction, identify
% the evidence, and forward-reference it from the claim. 
%==================================================================================================
\section{}

% The problem



%==================================================================================================
\section{}

% My idea



%==================================================================================================
\section{}

% The details

% Describe results carefully:
%  - clearly state assumptions
%  - give enough information to allow the reader to recreate the results
%  - ensure results are representative; statistically meaningful, etc.
%  - don't overstate results
%  - equally, don't understate them: consider the broader implications



%==================================================================================================
\section{Related Work}

% This should come near the end, and focussing on discussing how your work
% relates to that of others. Any relevant related work should have been
% cited already, so this is not a list of related work, it's a discussion
% of how that work relates.
%
% Why not put related work after the introduction? 1) because describing
% alternative approaches gets between the reader and your idea; and 2)
% because the reader knows nothing about the problem yet, so your
% (carefully trimmed) description of various technical trade-offs is
% absolutely incomprehensible.
% 
% When writing the related work:
%  - Give credit to others where it's due; this doesn't diminish the
%    credit you get from your paper. 
%  - Acknowledge weaknesses in your approach.
%  - Ensure related work is accurate and up-to-date



%==================================================================================================
\section{Conclusions}



%==================================================================================================
\section{Acknowledgements}

% Acknowledge funding sources.

%==================================================================================================
\bibliographystyle{abbrv}
\bibliography{paper}
\end{document}
% vim: set ts=2 sw=2 tw=75 et ai:
